%-------------------------
% https://www.overleaf.com/latex/templates/rezume/kfrvqywfkwjs
% Rezume, a latex resume template for developers
% Author : Nanu Panchamurthy
% Based off of: https://github.com/sb2nov/resume
% License : MIT

% Hope this resume template helps you land an awesome job. If you found this helpful, please consider starring the github repo here, .
%-------------------------



%------------PACKAGES----------------
\documentclass[a4paper,11pt]{article}

\usepackage{verbatim} % reimplements the "verbatim" and "verbatim*" environments

\usepackage{titlesec} % provides an interface to sectioning commands i.e. custom elements

\usepackage{color} % provides both foreground and background color management

\usepackage{enumitem} % provides control over enumerate, itemize and description

\usepackage{fancyhdr} % provides extensive facilities for constructing headers, footers and also controlling their use

\usepackage{tabularx} % defines an environment tabularx, extension of "tabular" with an extra designator x, paragraph like column whose width automatically expands to fill the width of the environment

\usepackage{latexsym} % provides mathematical symbols

\usepackage{marvosym} % provides martin vogel's symbol font which contains various symbols

\usepackage[empty]{fullpage} % sets margins to one inch and removes headers, footers etc..

\usepackage[hidelinks]{hyperref} % removes color and shadow of hyperlinks

\usepackage[normalem]{ulem} % provides "\ul" (uline) command which will break at line breaks

\usepackage[english]{babel} % provides culturally determined typographical rules for wide range of languages
%-----------------------------------------

\input glyphtounicode % converts glyph names to unicode
\pdfgentounicode=1 % ensures pdfs generated are ats readable

%----------FONT OPTIONS-------------------
\usepackage[default]{sourcesanspro} % uses the font source sans pro
\urlstyle{same} % changes url font from default urlfont to font being used by the document
%-----------------------------------------


%----------MARGIN OPTIONS-----------------
\pagestyle{fancy} % set page style to one configured by fancyhdr
\fancyhf{} % clear all header and footer fields

\renewcommand{\headrulewidth}{0in} % sets thickness of linerule under header to zero
\renewcommand{\footrulewidth}{0in} % sets thickness of linerule over footer to zero

\setlength{\tabcolsep}{0in} % sets thickness of column separator in tables to zero

% origin of the document is one inch from the top and from and the left
% oddsidemargin and evensidemargin both refer to the left margin
% right margin is indirectly set using oddsidemargin
\addtolength{\oddsidemargin}{-0.5in}
\addtolength{\topmargin}{-0.5in}

\addtolength{\textwidth}{1.0in} % sets width of text area in the page to one inch
\addtolength{\textheight}{1.0in} % sets height of text area in the page to one inch

\raggedbottom{} % makes all pages the height of current page, no extra vertical space added
\raggedright{} % makes all pages the width of current page, no extra horizontal space added
%------------------------------------------


%--------SECTIONING COMMANDS---------
% \titleformat{<command>}
%   [<shape>]{<format>}{<label>}{<sep>}
%     {<before-code>}[<after-code>]

% command is the sectioning command to be redefined
% shape is the style of the font; scshape stands for small caps style
% format is the format to be applied to whole title- label and text; absent here
% label defines the label
% sep is the horizontal separation between label and title body
% before-code is the code to be executed before
% after-code is the code to be executed after

\titleformat{\section}
{\scshape\large}{}
{0em}{\color{blue}}[\color{black}\titlerule\vspace{0pt}]
%-------------------------------------


%--------REDEFINITIONS----------------
% redefines the style of the bullet point
\renewcommand\labelitemii{$\vcenter{\hbox{\tiny$\bullet$}}$}

% redefines the underline depth to 2pt
\renewcommand{\ULdepth}{2pt}
%-------------------------------------


%--------CUSTOM COMMANDS--------------
%\vspace{} defines a vertical space of given size, modifying this in custom commands can help stretch or shrink resume to remove or add content

% resumeItem renders a bullet point
\newcommand{\resumeItem}[1]{
	\item\small{#1}
}

% commands to start and end itemization of resumeItem, rightmargin set to 0.11in to avoid the overflow of resumetItem beyond whatever resumeItemHeading is being used
\newcommand{\resumeItemListStart}{\begin{itemize}[rightmargin=0.11in]}
	\newcommand{\resumeItemListEnd}{\end{itemize}}

% resumeSectionType renders a bolded type to be used under a section, used as skill type here, middle element is used to keep ":"s in the same vertical line
\newcommand{\resumeSectionType}[3]{
	\item\begin{tabular*}{0.96\textwidth}[t]{
			p{0.15\linewidth}p{0.02\linewidth}p{0.81\linewidth}
		}
		\textbf{#1} & #2 & #3
	\end{tabular*}\vspace{-2pt}
}

% resumeTrioHeading renders three elements in three columns with second element being italicized and first element bolded, can be used for projects with three elements
\newcommand{\resumeTrioHeading}[3]{
	\item\small{
		\begin{tabular*}{0.96\textwidth}[t]{
				l@{\extracolsep{\fill}}c@{\extracolsep{\fill}}r
			}
			\textbf{#1} & \textit{#2} & #3
		\end{tabular*}
	}
}

% resumeQuadHeading renders four elements in a two columns with the second row being italicized and first element of first row bolded, can be used for experience and projects with four elements
\newcommand{\resumeQuadHeading}[4]{
	\item
	\begin{tabular*}{0.96\textwidth}[t]{l@{\extracolsep{\fill}}r}
		\textbf{#1} & #2 \\
		\textit{\small#3} & \textit{\small #4} \\
	\end{tabular*}
}

% resumeQuadHeadingChild renders the second row of resumeQuadHeading, can be used for experience if different roles in the same company need to added
\newcommand{\resumeQuadHeadingChild}[2]{
	\item
	\begin{tabular*}{0.96\textwidth}[t]{l@{\extracolsep{\fill}}r}
		\textbf{\small#1} & {\small#2} \\
	\end{tabular*}
}

% commands to start and end itemization of resumeQuadHeading, lefmargin for left indent of 0.15in for resumeItems
\newcommand{\resumeHeadingListStart}{
	\begin{itemize}[leftmargin=0.15in, label={}]
	}
	\newcommand{\resumeHeadingListEnd}{\end{itemize}}
%-------------------------------------------


%__________________RESUME____________________
% You can rearrange sections in any order you may prefer
\begin{document}
	
	%-----------CONTACT DETAILS------------------
	% Make sure all the details are correct, you can add more links in the first row of second column if needed
	
	\begin{tabular*}{\textwidth}{l@{\extracolsep{\fill}}r}
		\textbf{\Huge Daniel Precioso, PhD \vspace{2pt}} & % row = 1, col = 1
		Location: Madrid, Spain \\ % row = 1, col = 2
		\href{https://danielprecioso.com}{\uline{danielprecioso.com}} $|$ % row = 2, col = 1
		\href{https://www.linkedin.com/in/daniel-precioso-garcelan}{\uline{LinkedIn}} $|$ % row = 2, col = 1
		\href{https://github.com/daniprec}{\uline{GitHub}} % row = 2, col = 1
		$|$ \href{https://www.researchgate.net/profile/Daniel-Precioso-Garcelan}{\uline{ResearchGate}} % row = 2, col = 1
		& % row = 2, col = 1
		Email: \href{mailto:daniel.precioso@ie.edu}{\uline{daniel.precioso@ie.edu}} \\ % row = 2, col = 2
		& % row = 3, col = 1
		ORCID: \href{https://orcid.org/0000-0003-3836-1429}{\uline{0000-0003-3836-1429}} \\ % row = 3, col = 2
	\end{tabular*}
	%--------------------------------------------
	
	
	%-----------SUMMARY--------------------------
	% Keep this short, simple and straigth to point
	
	\section{Data Scientist}
	% \small{
		 I have \textbf{over 6 years of experience} in \textbf{machine learning} and \textbf{Python}. My journey has taken me through both academic research and industry roles, where I participated in projects related to healthcare, energy, and logistics. My main skills include problem-solving, automating tasks and simplifying complex ideas.
		 
		 I am passionate about collaborating with others, mentoring teammates, and teaching. One of my favourite projects involved developing the ``Google Maps of the Sea'', using mathematical optimization and weather data to reduce the GHG emissions of shipping routes. I was invited to \href{https://www.youtube.com/watch?v=T2VRDSczxVM&t}{\uline{pitch this idea to the European Parliament}}!
		% }
	%--------------------------------------------
	
	
	%--------------SKILLS------------------------
	% Add or remove resumeSectionTypes according to your needs
	
	\section{Technical Skills}
	\resumeHeadingListStart{}
	\resumeSectionType{Languages}{:}{Python, Matlab, R}
	\resumeSectionType{Libraries}{:}{NumPy, Pandas, scikit-learn, TensorFlow, JAX, Streamlit}
	\resumeSectionType{Dev Tools}{:}{Visual Studio Code, Git, Github}
	\resumeSectionType{Academic Tools}{:}{Latex, Overleaf, Blackboard, Wordpress}
	\resumeHeadingListEnd{}
	%--------------------------------------------
	
	
	%-----------EXPERIENCE-----------------------
	% Distill all your talking points to small bullet points which follow the pattern "challenge-action-result" for maximum efficiency. Try to quantify (use numbers) your points whenver possible, highlist words of importance
	
	\section{Experience}
	\resumeHeadingListStart{}
	
	\resumeQuadHeading{Postdoctoral Researcher \& Adjunct Professor}{Sep 2023 -- Present}
	{IE University}{Madrid, Spain}
	\resumeItemListStart{}
	\resumeItem{Conducting research on mathematical optimization to reduce GHG emissions of ships using weather data.}
	\resumeItem{Active member of IE Research Datalab; responsible for writing and designing the official website.}
	\resumeItem{Teaching courses in the Bachelor of Applied Mathematics program: Computer Programming I, Coding Lab, and Applied Math Lab.}
	\resumeItemListEnd{}
	
	\resumeQuadHeading{Director of Research and Development}{Apr 2023 -- Present}
	{Canonical Green}{Madrid, Spain}
	\resumeItemListStart{}
	\resumeItem{Developing data science solutions for ecological transition in the maritime industry.}
	\resumeItem{Delivering persuasive commercial and technical presentations.}
	\resumeItemListEnd{}
	
	\resumeQuadHeading{Data Scientist}{Sep 2022 -- Apr 2023}
	{Komorebi AI}{Madrid, Spain}
	\resumeItemListStart{}
	\resumeItem{Performing data cleaning, manipulation, and visualization.}
	\resumeItem{Designing, training, and deploying machine learning and deep learning models using scikit-learn.}
	\resumeItem{Developing a dashboard to guide industrial decision-making using Streamlit.}
	\resumeItemListEnd{}
	
	\resumeQuadHeading{Predoctoral Research Staff}{Sep 2019 -- Aug 2022}
	{University of C\'adiz}{C\'adiz, Spain}
	\resumeItemListStart{}
	\resumeItem{Applying machine learning for Industry 4.0: non-intrusive load monitoring and analysis of fishing populations.}
	\resumeItem{Utilizing data science in healthcare: forecasting ICU use during COVID and monitoring neonates using computer vision.}
	\resumeItem{Presenting research findings to both technical and non-technical audiences.}
	\resumeItem{Publishing research papers in peer-reviewed journals.}
	\resumeItemListEnd{}
	
	\resumeQuadHeading{Junior Data Scientist}{Jan 2019 -- Jun 2019}{Foqum}{Madrid, Spain}
	
	\resumeHeadingListEnd{}
	%---------------------------------------------
	
	
	%-----------EDUCATION-------------------------
	% Mention your CGPA, if its good, in the first row of second column
	
	\section{Education}
	\resumeHeadingListStart{}
	
	\resumeQuadHeading{Phd in Data Science}{Sep 2019 -- Jul 2023}
	{University of C\'adiz}{C\'adiz, Spain}
	
	\resumeQuadHeading{MSc in Statistical and Computational Information Processing}{Sep 2018 -- Jul 2019}
	{Universidad Polit\'ecnica de Madrid}{Madrid, Spain}
	
	\resumeQuadHeading{Degree in Physics}{Sep 2014 -- Jul 2018}
	{Complutense University of Madrid}{Madrid, Spain}
	
	\resumeHeadingListEnd{}
	%---------------------------------------------
	
	
	%-----------PROJECTS--------------------------
	% Use resumeQuadHeading if four elements are feasible (ex: demo video link), else use resumeTrioHeading. Keep the bullet points simple and concise and try to cover wide variety of skills you have used to build these projects
	
	\section{Projects}
	\resumeHeadingListStart{}
	
	\resumeTrioHeading{Weather Navigation}{Mathematical Optimization, Maritime Transport}{IE University}
	\resumeItemListStart{}
	\resumeItem{Optimization of maritime routes for a more efficient, safer and decarbonized transport}
	\resumeItem{Funded by the BBVA Foundation and the Spanish Agencia Estatal de Investigación under grant TED2021-129455B-I00.}
	\resumeItem{I served as a main researcher and maintained the official project website \href{https://weathernavigation.com/}{\uline{https://weathernavigation.com/}}.}
	\resumeItemListEnd{}
	
	\resumeTrioHeading{NeoCam}{Computer Vision, Healthcare}{Univesidad de C\'adiz}
	\resumeItemListStart{}
	\resumeItem{We used the Luxonis OAK-D smart camera to build a contactless monitoring system for newborn babies.}
	\resumeItem{The proposed solution combined computer vision, machine learning, edge computing, cloud computing and Internet of things.}
	\resumeItem{NeoCam project was awarded the second prize in the international final of \href{https://opencv.org/opencv-ai-competition-2021/\#regional-winners}{\uline{OpenCV AI Competition 2021}}, in which over 1400 teams participated.}
	\resumeItemListEnd{}
	
	\resumeTrioHeading{Smart Shipping}{Mathematical Optimization, Maritime Transport}{Univesidad de C\'adiz}
	\resumeItemListStart{}
	\resumeItem{Our goal was to optimize marine shipping routes, by using real time information of ocean currents, wind and waves.}
	\resumeItem{Smart Shipping project was awarded the second prize in the international final of \href{https://www.youtube.com/watch?v=nzPbj88IV0c&t=5300s}{\uline{Ocean Hackathon 2021}}, hosted by Campus Mondiale de la Mer in Brest (France).}
	\resumeItemListEnd{}
	
	\resumeTrioHeading{UCAnFly}{Computer Modeling, Astrophysics}{Univesidad de C\'adiz}
	\resumeItemListStart{}
	\resumeItem{We designed an nanosatellite to test emerging technologies for space-based gravitational wave detectors, such as \href{https://www.elisascience.org/}{\uline{LISA}}.}
	\resumeItem{UCAnFly was led by a multidisciplinary team at the University of C\'adiz, with the support of the Education Office of the European Space Agency, under the educational \href{https://www.esa.int/Education/CubeSats_-_Fly_Your_Satellite}{\uline{Fly Your Satellite!}} programme.}
	\resumeItemListEnd{}
	
	\resumeTrioHeading{ATENEA}{Computer Vision, Industrial Automation}{Univesidad de C\'adiz}
	\resumeItemListStart{}
	\resumeItem{This project was hosted by Airbus D\&S with financial support from the CDTI Interconnecta program.}
	\resumeItem{The aim was to introduce machine learning and natural language processing to streamline certain manufacturing processes in Airbus D\&S production plants.}
	\resumeItemListEnd{}
	
	\resumeHeadingListEnd{}
	%--------------------------------------------
	
	
	%----------------OTHERS----------------------
	% You can add your acheivements, accolades, certifications etc. here.
	
	\section{Certifications}
	\resumeItemListStart{}
	\resumeItem{\href{https://openbadgefactory.com/v1/assertion/4d76a8035a509add2f9b01f1c19cdb9a0eef56ca.html}{\uline{Machine learning in Python with scikit-learn}} (France Universit\'e Num\'erique)}
	
	\resumeItem{\href{https://drive.google.com/file/d/1hNx8BzQxnWpydmgg9RoCtc44FBE6HSEG/view?usp=sharing}{\uline{XV Modeling Week at UCM (Coordinator)}} (UCM)}
	
	\resumeItem{\href{https://drive.google.com/file/d/1GrqmPLSJoitdC7otmq_SRfqXVbEOuW_F/view?usp=sharing}{\uline{Fly your Satellite - 3 CDR Virtual Workshop}} (ESA)}
	
	\resumeItem{\href{https://www.coursera.org/account/accomplishments/specialization/WXQQSYVFLA78}{\uline{TensorFlow in Practice Specialization}} (Coursera)}
	%\resumeItem{\href{https://www.coursera.org/account/accomplishments/verify/FE4HNB2LFVRU}{\uline{Convolutional Neural Networks in TensorFlow}} (Coursera)}
	%\resumeItem{\href{https://www.coursera.org/account/accomplishments/verify/XG8F2HMXUH4V}{\uline{Introduction to TensorFlow for Artificial Intelligence, Machine Learning, and Deep Learning}} (Coursera)}
	%\resumeItem{\href{https://www.coursera.org/account/accomplishments/verify/FPHZBNAU29ZK}{\uline{Natural Language Processing in TensorFlow}} (Coursera)}
	%\resumeItem{\href{https://www.coursera.org/account/accomplishments/verify/DKX6PA6UFBQC}{\uline{Sequences, Time Series and Prediction}} (Coursera)}
	
	\resumeItem{\href{https://www.coursera.org/account/accomplishments/verify/JEZRF6VKSWUE}{\uline{Applied Social Network Analysis in Python}} (Coursera)}
	\resumeItem{\href{https://www.coursera.org/account/accomplishments/verify/HKSAYBJFPQGU}{\uline{Applied Machine Learning in Python}} (Coursera)}
	\resumeItem{\href{https://www.coursera.org/account/accomplishments/verify/PMFZNFG755X3}{\uline{Applied Text Mining in Python}} (Coursera)}
	\resumeItem{\href{https://www.coursera.org/account/accomplishments/verify/2SDFQS2HSAE3}{\uline{Introduction to Data Science in Python}} (Coursera)}
	
	\resumeItemListEnd{}
	
	\section{Publications}
	\resumeItemListStart{}
	\resumeItem{\href{https://benchmark.weathernavigation.com/static/pdfs/benchmarks-1.pdf}{\uline{HADAD: Hexagonal A-Star with Differential Algorithm for Data-driven routing}}}. In review
	\resumeItem{\href{https://doi.org/10.1007/s40314-024-02756-w}{\uline{Hybrid Search method for Zermelo's navigation problem}}}. Computational and Applied Mathematics, 2024 (Q2)
	\resumeItem{\href{https://doi.org/10.3354/meps14338}{\uline{Aggregation dynamics of tropical tunas around drifting floating objects based on large-scale echo-sounder data}}}. Marine Ecology Progress Series, 2023 (Q1)
	\resumeItem{\href{https://doi.org/10.21203/rs.3.rs-1732801/v1}{\uline{Effectiveness of non-pharmaceutical interventions in nine fields of activity to decrease SARS-CoV-2 transmission (Spain, September 2020-May 2021)}}}. Frontiers in Public Health, 2023 (Q1)
	\resumeItem{\href{https://doi.org/10.21203/rs.3.rs-1923023/v1}{\uline{Thresholding Methods in Non-Intrusive Load Monitoring}}}. The Journal of Supercomputing, 2023 (Q2)
	\resumeItem{\href{https://doi.org/10.1109/JBHI.2023.3240245}{\uline{NeoCam: An edge-cloud platform for non-invasive real-time monitoring in neonatal intensive care units}}}. IEEE Journal of Biomedical and Health Informatics, 2023 (Q1)
	\resumeItem{\href{https://doi.org/10.1016/j.fishres.2022.106263}{\uline{TUN-AI: Tuna biomass estimation with Machine Learning models trained on oceanography and echosounder FAD data}}}. Fisheries Research, 2022 (Q2)
	%\resumeItem{\href{}{\uline{}}}
	\resumeItemListEnd{}
	%--------------------------------------------
	
\end{document}